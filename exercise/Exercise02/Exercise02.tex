\documentclass[12pt,a4paper]{article}
\usepackage{listings}
\usepackage{xcolor}
\lstset{
	columns=fixed,       
	numbers=left,                                        % 在左侧显示行号
	frame=none,                                          % 不显示背景边框
	backgroundcolor=\color[RGB]{245,245,244},            % 设定背景颜色
	keywordstyle=\color[RGB]{40,40,255},                 % 设定关键字颜色
	numberstyle=\footnotesize\color{darkgray},           % 设定行号格式
	commentstyle=\it\color[RGB]{0,96,96},                % 设置代码注释的格式
	stringstyle=\rmfamily\slshape\color[RGB]{128,0,0},   % 设置字符串格式
	showstringspaces=false,                              % 不显示字符串中的空格
	language=c++,										 % 设置语言
	morekeywords={NULL, Semaphore}                                       
}
\begin{document}
	\begin{enumerate}
		\item In the three given schemes, I would choose the second one, that is, use a mutex over each hash bin. The reason is that the first one would only allow one thread to access the hash table at any given time, thus decreasing the performance dramatically. The second one would allow multiple threads to access the hash table at the same time, but if there are threads want to modify a hash bin at the same time, the mutex will guarantee these modifications executed in a well ordered way. The third one is like the second one, but would be harder to implement. Thus, I would choose the second one.
		
		\item The given implementation is not correct. There are several reasons, first, if $q2$ is empty, it fails to return $q1$ the same as befofe the swap was attempted. Second, if $q1$ or $q2$ is empty, the lock of them will be grabbed but not  released, resulting in the following accesses to these two queues impossible. Third, a deadlock would happen, if atomic\_swap(q1,q2) and atomic\_swap(q2,q1) are called at almost the same time.A possible modification of the given implementation could be like below.
{
\begin{lstlisting}
extern Item* dequeue(Queue*); 
// pops an item from a stack
extern void enqueue(Queue*, Item*); 
// pushes an item onto a stack
void atomic_swap(Queue* q1, Queue* q2)
{
	Item* item1;
	Item* item2; // items being transferred 
	sort the two argumeents in some way;
	// so when atomic_swap(q1, q2) and 
	// atomic_swap(q2, q1) are called
	// at the same time, there would not be a 
	//deadlock
	wait(q1->lock);
	wait(q2->lock);
	item1 = pop(q1);
	if (item1 != NULL)
	{
		item2 = pop(q2);
		if (item2 != NULL)
		{
			push(q2, item1);
			push(q1, item2);
		}
		else
		{
			push(q1, item1);
		}
	}
	singal(q2->lock);
	singal(q1->lock);
}
\end{lstlisting}
}
		\item 
			\begin{enumerate}
				\item `smokers' is used to indicate the number of smokers who are smoking now and `nonsmokers' is used to indicate the number of nonsmokers in the lounge.
				\begin{lstlisting}
int smokers = 0, nonsmokers = 0;
				\end{lstlisting}
				
				\item `enter' is used to guarantee the executing order is the same as the arriving order of people. `mutexSmoker' is used to protect the `smokers' and `mutexNonsmoker' is used to protect the `nonsmokers'.
				\begin{lstlisting}
Semaphore mutexSmoker = mutexNonsmoker = 1;
Semaphore enter = smoking = 1;
				\end{lstlisting}
				
				\item One possible solution is like below.
				\begin{lstlisting}
enterlounge(true)
{
	wait(enter);
	signal(enter);
}

enterlounge(false)
{
	wait(enter);
	wait(mutexNonSmoker);
	if (nonsmokers == 0)
		wait(smoking);
	nonSmokers++;
	signal(mutexNonSmoker);
	signal(enter);
}

smoke()
{
	wait(mutexSmoker)
	if (smokers == 0)
		wait(smoking);
	smokers++;
	signal(mutexSmoker);
	// smoking here;
	wait(mutexSmoker);
	smokers--;
	If(smokers == 0)
		signal(smoking);
	signal(mutexSmoker);
}

leaveLounge(true)
{
 //just leave
}

leaveLounge(false)
{
	wait(mutexNonSmoker);
	nonSmokers--;
	if (nonSmokers == 0)
		signal(smoking);
	signal(mutexNonSmoker);
}
				\end{lstlisting}
				\item \begin{itemize}
					\item Mutual exclusion. If process $P_i$ is executing in its critical section, then no other processes can be executing in their critical sections.
					\item Progress. If no process is executing in its critical section and there exist some processes that wish to enter their critical section, then the selection of the processes that will enter the critical section next cannot be postponed indefinitely.
					\item Bounded waiting. A bound must exist on the number of times that other
					processes are allowed to enter their critical sections after a process has
					made a request to enter its critical section and before that request is
					granted
					\item Sequential Access. The sequence of accessing the critical section
					follows the order of the requests raised by the processes
				\end{itemize}
				The `mutexSmoker', `mutexNonsmoker' and `smoking' semaphores guarantee the first three properties, and the `enter' semaphore guarantees the last propertie.
			\end{enumerate}
	\end{enumerate}
\end{document}